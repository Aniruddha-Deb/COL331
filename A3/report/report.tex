\documentclass[12pt]{article}
\usepackage[a4paper,margin=0.75in]{geometry}
\usepackage[utf8]{inputenc}
\usepackage[OT1]{fontenc}
\usepackage[table,usenames,dvipsnames]{xcolor}
\usepackage{array}
\usepackage{varwidth}
\usepackage{tabularx}
\usepackage{amsmath}
\usepackage{hyperref}
\usepackage{enumitem}
\usepackage{graphicx}
\usepackage{tcolorbox}
\usepackage{forest}
\usepackage{parskip}
\renewcommand*\familydefault{\sfdefault}

% from https://gist.github.com/nhtranngoc/88b72d9bfb656a3de227eea38ed80627
\usepackage{listings}
% \usepackage{fontspec}
% \setmonofont{Consolas}

\definecolor{background}{RGB}{39, 40, 34}
\definecolor{string}{RGB}{230, 219, 116}
\definecolor{comment}{RGB}{117, 113, 94}
\definecolor{normal}{RGB}{248, 248, 242}
\definecolor{identifier}{RGB}{166, 226, 46}

\lstset{
  % language=c,                			% choose the language of the code
  numbers=left,                   		% where to put the line-numbers
  stepnumber=1,                   		% the step between two line-numbers.        
  numbersep=5pt,                  		% how far the line-numbers are from the code
  numberstyle=\scriptsize\color{black}\ttfamily,
  backgroundcolor=\color{background},  		% choose the background color. You must add \usepackage{color}
  showspaces=false,               		% show spaces adding particular underscores
  showstringspaces=false,         		% underline spaces within strings
  showtabs=false,                 		% show tabs within strings adding particular underscores
  tabsize=4,                      		% sets default tabsize to 2 spaces
  captionpos=b,                   		% sets the caption-position to bottom
  breaklines=true,                		% sets automatic line breaking
  breakatwhitespace=true,         		% sets if automatic breaks should only happen at whitespace
  title=\lstname,                 		% show the filename of files included with \lstinputlisting;
  basicstyle=\color{normal}\ttfamily\footnotesize,	    	% sets font style for the code
  keywordstyle=\color{magenta}\ttfamily,	% sets color for keywords
  stringstyle=\color{string}\ttfamily,		% sets color for strings
  commentstyle=\color{comment}\ttfamily,	% sets color for comments
  emph={format_string, eff_ana_bf, permute, eff_ana_btr},
  emphstyle=\color{identifier}\ttfamily
}

\newtcolorbox{mybox}[3][]
{
  colframe = #2!25,
  colback  = #2!10,
  coltitle = #2!20!black,  
  title    = {#3},
  #1,
}

\hypersetup{
    colorlinks=true,
    linkcolor=blue,
    filecolor=magenta,      
    urlcolor=cyan,
    pdftitle={Overleaf Example},
    pdfpagemode=FullScreen,
}

\title{\textbf{COL331 Assignment 3: Hard Track}}
\author{Aniruddha Deb \\ \texttt{2020CS10869}}
\date{April 2023}

\begin{document}

\maketitle

\section{Implementation Overview}

The implementation involves creating a single device driver which initializes 
two devices, a reader, located at \texttt{/dev/lifo\_reader} and a writer, 
located at \texttt{/dev/lifo\_writer}. Writing to the reader is disallowed, and
reading from the writer is disallowed. The devices have read and write Permissions 
for all users. 

Internally, we use a single structure with two device drivers, a 1MB buffer and
a pointer to the head of this buffer to implement the driver. The writes reverse
the order of the string while writing to this buffer (thereby making it LIFO),
and the reads read in the order in the buffer. There is explicit synchronization
to prevent race conditions, and two separate \texttt{file\_operations} structures
are created for the reader and the writer, both supporting different operations.
The main struct for storing the driver state is shown below:

\begin{lstlisting}[language=C]
struct lifo_cdev_data {
	struct cdev reader;
	struct cdev writer;
	// in-memory 1MB buffer
	char* buffer;
	ssize_t head;

	// synchronization
	struct semaphore sem;
	uid_t owner;
	pid_t process;
};
\end{lstlisting}

\section{Implementation Details}

\subsection{Module access rules}

Only one user can have ownership of either kernel module at a given time, and 
only one process may have ownership of either the reader device or the writer
device at a given time. This is to prevent race conditions that may lead to
disorder in the stream

Therefore, we have the following rules:
\begin{enumerate}
\item Only one process can access either the lifo reader or lifo writer at any given instant
\item If another process tries to access the lifo reader or lifo writer when they're in use:
\begin{enumerate}
    \item If they're from another user, the request is rejected (we return -EBUSY) till all the 
      tasks of the current owner are done
  \item If they're from the same user, the request blocks till the previous accesses are over.
\end{enumerate}
\item If the same process tries to both read and write from the device, then
since the reads block till the writes are over (and vice versa), the process
may go into deadlock, since it may wait for a write and itself be unable to write
as it's waiting. Hence, we prevent this by returning -EINVAL if this occurs.
\end{enumerate}

The relevant code for this is as follows:

\begin{lstlisting}[language=C]
int __acquire_lifo_semaphore(void) {
	if (lifo_cdev.owner != current->loginuid.val && lifo_cdev.owner != 0) return -EBUSY;
	if (lifo_cdev.process == current->pid) return -EINVAL;
	down(&lifo_cdev.sem);
	lifo_cdev.owner = current->loginuid.val;
	lifo_cdev.process = current->pid;
	return 0;
}
\end{lstlisting}

\subsection{Creating device files}

To create a device file, a device region needs to be allocated first. This 
corresponds to a major number, and a range of minor numbers. We let the OS 
allocate us a major number, rather than explicitly setting it ourselves to 
avoid errors. This is done using \texttt{alloc\_chrdev\_region}. After this, the
device class has to be created using the major number. Device files are then 
created using the \texttt{cdev\_init} function and added to the OS using \texttt{cdev\_add}.

After this, we allocate 1MB of memory for the kernel buffer and initialize the 
cdev semaphore, and set the devnode function to ensure that appropriate Permissions
are set for the devices that have been created.

\subsection{Preventing reads (writes) from writer (reader)}

This is implemented by having the respective methods return an error code (-EINVAL)
when someone attempts to run them. The Operating System takes care of the invalid 
return code and makes it unable to read/write from these files.

\begin{lstlisting}[language=C]
$ cat /dev/lifo_writer 
cat: /dev/lifo_writer: Invalid argument
\end{lstlisting}

\subsection{Compiling as a kernel module}

To compile as a kernel module, the driver file was placed in the \texttt{drivers/char}
folder, and a driver description was written in the \texttt{drivers/char/Kconfig}
file for this driver. After this, the driver was enabled in the menuconfig and
the kernel was compiled with this driver.

\subsection{Other details}

EOF was not explicitly set when there are no writers, as the EOF character is
not 0x05 but is implementation defined. It is also error-prone to encode a byte 
as a termination character in a byte stream as the probability that the byte 
stream itself contains that character will be high, and may lead to premature 
termination (especially if we're using the byte stream to encode multiple-byte 
wide characters. 

EOF is detected by utilities eg. cat by returning 0 from the \texttt{reader\_read}
method when there are no bytes to read. In other cases, the buffer is zeroed 
by using the \texttt{clear\_user} method.

\section{Testing}

Elementary testing was done by simple file-based I/O utilities in bash 

\begin{lstlisting}[language=C]
$ printf "Hello" > /dev/lifo_writer
$ head -c 5 /dev/lifo_reader
olleH$
\end{lstlisting}

For writing an arbitrary number of bytes, a C++ program was written with the following 
pseudocode:

\begin{lstlisting}[language=C++]
FILE *lifo_writer = fopen("/dev/lifo_writer", "wb");
int n_bytes;
while (1) {
        std::cin >> n_bytes;
        if (n_bytes == 0) break;

        char *buf = (char*)malloc(n_bytes);
        for (int i=0; i<n_bytes; i++) buf[i] = (char)i;
        
        int n_bytes_written = fwrite(buf, 1, n_bytes, lifo_writer);
        fflush(lifo_writer);
}
\end{lstlisting}

A reader was simultaneously used (which could read an arbitrary no. of bytes)
to test the various access cases described above: the reader was simply
\texttt{head -c <no. characters>}. \texttt{dmesg} was used to check the logs to
see if the program was working correctly.

\end{document}
